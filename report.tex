% Options for packages loaded elsewhere
\PassOptionsToPackage{unicode}{hyperref}
\PassOptionsToPackage{hyphens}{url}
%
\documentclass[
]{article}
\usepackage{amsmath,amssymb}
\usepackage{iftex}
\ifPDFTeX
  \usepackage[T1]{fontenc}
  \usepackage[utf8]{inputenc}
  \usepackage{textcomp} % provide euro and other symbols
\else % if luatex or xetex
  \usepackage{unicode-math} % this also loads fontspec
  \defaultfontfeatures{Scale=MatchLowercase}
  \defaultfontfeatures[\rmfamily]{Ligatures=TeX,Scale=1}
\fi
\usepackage{lmodern}
\ifPDFTeX\else
  % xetex/luatex font selection
\fi
% Use upquote if available, for straight quotes in verbatim environments
\IfFileExists{upquote.sty}{\usepackage{upquote}}{}
\IfFileExists{microtype.sty}{% use microtype if available
  \usepackage[]{microtype}
  \UseMicrotypeSet[protrusion]{basicmath} % disable protrusion for tt fonts
}{}
\makeatletter
\@ifundefined{KOMAClassName}{% if non-KOMA class
  \IfFileExists{parskip.sty}{%
    \usepackage{parskip}
  }{% else
    \setlength{\parindent}{0pt}
    \setlength{\parskip}{6pt plus 2pt minus 1pt}}
}{% if KOMA class
  \KOMAoptions{parskip=half}}
\makeatother
\usepackage{xcolor}
\usepackage[margin=1in]{geometry}
\usepackage{graphicx}
\makeatletter
\def\maxwidth{\ifdim\Gin@nat@width>\linewidth\linewidth\else\Gin@nat@width\fi}
\def\maxheight{\ifdim\Gin@nat@height>\textheight\textheight\else\Gin@nat@height\fi}
\makeatother
% Scale images if necessary, so that they will not overflow the page
% margins by default, and it is still possible to overwrite the defaults
% using explicit options in \includegraphics[width, height, ...]{}
\setkeys{Gin}{width=\maxwidth,height=\maxheight,keepaspectratio}
% Set default figure placement to htbp
\makeatletter
\def\fps@figure{htbp}
\makeatother
\setlength{\emergencystretch}{3em} % prevent overfull lines
\providecommand{\tightlist}{%
  \setlength{\itemsep}{0pt}\setlength{\parskip}{0pt}}
\setcounter{secnumdepth}{-\maxdimen} % remove section numbering
\ifLuaTeX
  \usepackage{selnolig}  % disable illegal ligatures
\fi
\usepackage{bookmark}
\IfFileExists{xurl.sty}{\usepackage{xurl}}{} % add URL line breaks if available
\urlstyle{same}
\hypersetup{
  pdftitle={CITS4009 Project: Exploratory Data Analytics and Predictive Modelling},
  pdfauthor={Dayu LIU (24188516)},
  hidelinks,
  pdfcreator={LaTeX via pandoc}}

\title{CITS4009 Project: Exploratory Data Analytics and Predictive
Modelling}
\author{Dayu LIU (24188516)}
\date{Semester 2, 2024}

\begin{document}
\maketitle

\section{Introduction}\label{introduction}

In this project, we are going to analyze the dataset from WHO, study the
dataset, make a prediction and make a model to demonstrate it
\vspace{0.5cm}

\section{Data exploration}\label{data-exploration}

In this first step of our project, we will first have a glance at the
dataset. We begin by exploring the structure and summary of the dataset
to understand its composition and the variables it contains. From
\texttt{str(df)}, we can see that the dataset contains 6,840 rows and 31
variables. Each row represents data for a specific country and year,
while the columns consist of a wide range of health and environmental
factors contributing to death rates. Some of the Key columns include:
Entity: Country name. Year: Data collection year, ranging from 1990 to
2019. Outdoor Air Pollution: Air pollution data measured in number of
deaths. High Systolic Blood Pressure: Health risk factor measured by the
number of deaths. From \texttt{summary(df)}, we can observe that we
observe that: Outdoor Air Pollution: The minimum value is 0, while the
maximum is 4,506,193 deaths. The median value is much lower, indicating
that only a few countries have very high deaths due to air pollution.
High Systolic Blood Pressure: The values range widely, with a median of
8,770 deaths and a maximum of over 10 million. This shows it's a
significant cause of death globally. Other risk factors, such as Alcohol
Use, Smoking, and Unsafe Water Source, show significant variation across
the countries. From \texttt{head(df)}, The first six rows of the dataset
show data for Afghanistan from 1990 to 1995. A few key trends:

Outdoor Air Pollution: A slight increase in deaths over time, from 3,169
in 1990 to 3,869 in 1995. High Systolic Blood Pressure: The number of
deaths rises consistently, from 25,633 in 1990 to 28,090 in 1995. Child
Stunting: A concerning rise in child stunting from 7,686 in 1990 to
11,973 in 1995. This analysis shows that health risks and causes of
death vary significantly across years, making it a crucial dataset for
public health planning.

\end{document}
